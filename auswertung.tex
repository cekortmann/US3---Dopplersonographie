\section{Auswertung}
\label{sec:Auswertung}

\subsection{Fehlerrechnung}
\label{sec:Fehlerrechnung}
Für die Fehlerrechnung werden folgende Formeln aus der Vorlesung verwendet.
für den Mittelwert gilt
\begin{equation}
    \overline{x}=\frac{1}{N}\sum_{i=1}^N x_i ß\; \;\text{mit der Anzahl N und den Messwerten x} 
    \label{eqn:Mittelwert}
\end{equation}
Der Fehler für den Mittelwert lässt sich gemäß
\begin{equation}
    \increment \overline{x}=\frac{1}{\sqrt{N}}\sqrt{\frac{1}{N-1}\sum_{i=1}^N(x_i-\overline{x})^2}
    \label{eqn:FehlerMittelwert}
\end{equation}
berechnen.
Wenn im weiteren Verlauf der Berechnung mit der fehlerhaften Größe gerechnet wird, kann der Fehler der folgenden Größe
mittels Gaußscher Fehlerfortpflanzung berechnet werden. Die Formel hierfür ist
\begin{equation}
    \increment f= \sqrt{\sum_{i=1}^N\left(\frac{\partial f}{\partial x_i}\right)^2\cdot(\increment x_i)^2}.
    \label{eqn:GaussMittelwert}
\end{equation}
\subsection{Auswertung der Messwerte}
Zunächst müssen die Dopplerwinkel $\alpha$ für die Winkel $\theta$ berechnet werden.
Die \autoref{eqn:alpha} liefert so 
\begin{align*}
     \alpha(\theta =15°) &= 80.06\\
     \alpha(\theta =30°) &=70.53\\
     \alpha(\theta =45°) &=54.74\\
\end{align*}
\subsubsection{Bestimmung der Strömungsgeschwindigkeit}
\label{sec:Strömungsv}
Im ersten Teil der Auswertung wird die Strömungsgeschwindigkeit bestimmt. Die hierfür nötigen Messwerte sind in \autoref{tab:Stroemungsv} gelistet.
\begin{table}
    \centering
    \caption{Messwerte zu Bestimmung der Strömungsgeschwindigkeit.}
    \begin{tabular}{c c c c c c}
        \toprule
        Rpm der Pumpe  & Winkel $\theta \mathrm{/} \unit{\degree}$ & Winkel $\alpha \mathrm{/}  \unit{\degree}$ & $\nu_{\text{max}} \mathrm{/} \unit{\hertz}$ & $\nu_{\text{mean}} \mathrm{/} \unit{\hertz}$ &  $ \Delta\numathrm{/} \unit{\hertz}$ \\
        \midrule
        2990 & 15 & 80.06 & 175 & 85 & 90  \\
             & 30 & 70.53 & 210 & 110 & 100 \\
             & 60 & 54.74 & 303 & 159 & 144 \\
        4020 & 15 & 80.06 & 290 & 146 & 144 \\
             & 30 & 70.53 & 360 & 171 & 189 \\
             & 60 & 54.74 & 570 & 370 & 200 \\
        5010 & 15 & 80.06 & 380 & 183 & 197 \\
             & 30 & 70.53 & 565 & 269 & 296 \\
             & 60 & 54.74 & 960 & 513 & 447 \\
        6000 & 15 & 80.06 & 605 & 256 & 349 \\
             & 30 & 70.53 & 850 & 403 & 447 \\
             & 60 & 54.74 & 1505 & 757 & 748 \\
        7000 & 15 & 80.06 & 850 & 378 & 472  \\
             & 30 & 70.53 & 1160 & 519 & 641 \\
             & 60 & 54.74 & 2020 & 1013 & 1007 \\
        \bottomrule
    \end{tabular}
    \label{tab:Stroemungsv}
\end{table}
Eine Formel für dei Flussgeschwindigkeit ergibt sich durch das Umstellen von \autoref{eqn:dv} nach $v$
\begin{equation*}
v=\frac{c\cdot \Delta \nu}{2\nu_0\cdot \cos \alpha}\. .
\end{equation*}
So ergeben sich die folgenden Flussgeschwindigkeiten.
\begin{table}
     \centering
     \caption{Fließgeschwindigkeiten der verschiedenen Winkel und Pumpleistungen.}
     \begin{tabular}{c c c c c c}
          \toprule
          Rpm der Pumpe & $v_{15°}\mathrm{/} \unit{\frac{\meter}{\second}}$ & $v_{30°}\mathrm{/} \unit{\frac{\meter}{\second}}$& $v_{60°}\mathrm{/} \unit{\frac{\meter}{\second}}$ & $\bar{v}\mathrm{/} \unit{\frac{\meter}{\second}}$\\
          \midrule
          2990 & 0.2346 & 0.1350 & 0.1122 & 0.1605\\
          4020 & 0.3754 & 0.2552 & 0.1559 & 0.2621\\
          5010 & 0.5136 & 0.3996 & 0.3484 & 0.4205\\
          6000 & 0.9098 & 0.6035 & 0.5831 & 0.6988\\
          7000 & 1.2305 & 0.8654 & 0.7850 & 0.9603\\ 
          \bottomrule
     \end{tabular}
\end{table}
In der letzten Spalte stehen die Mittelwerte nach \autoref{eqn:Mittelwert} der Geschwindigkeiten zu den einzelnen Drehzahlen der Pumpe. 
Neben der Berechnung der Geschwindigkeiten $v$ wird auch der Quotient der Frequenzverschiebung $\Delta \nu$ und der Kosinus des Dopplerwinkels $\alpha$ gegen die Geschwindigkeit $v$ aufgetragen.
Diese Graphen sind in \autoref{fig:Winkel15}, \autoref{fig:Winkel30} und \autoref{fig:Winkel45} dargestellt.
\begin{figure}
     \centering
     \includegraphics[height = 6cm]{build/plot11.pdf}
     \caption{Auftragung des Quotienten der Frequenzverschiebung $\Delta \nu$ und des Kosinus des Dopplerwinkels $\alpha=80.06°$ gegen die Geschwindigkeit $v$.}
     \label{fig:Winkel15}
\end{figure}

\begin{figure}
     \centering
     \includegraphics[height = 6cm]{build/plot12.pdf}
     \caption{Auftragung des Quotienten der Frequenzverschiebung $\Delta \nu$ und des Kosinus des Dopplerwinkels $\alpha=70.53°$ gegen die Geschwindigkeit $v$.}
     \label{fig:Winkel30}
\end{figure}

\begin{figure}
     \centering
     \includegraphics[height = 6cm]{build/plot13.pdf}
     \caption{Auftragung des Quotienten der Frequenzverschiebung $\Delta \nu$ und des Kosinus des Dopplerwinkels $\alpha=54.74°$ gegen die Geschwindigkeit $v$.}
     \label{fig:Winkel45}
\end{figure}

Aufgrund der Tatsache, dass die Pumpe einen Defekt aufwies, welcher ein Anzeigen des Volumenstroms verhinderte, könne Theoretische Werte nur mit dem Abgleichen der maximalen Pumpleistung mit jener einer anderen Pumpe verglichen werden.
Die maximale Drehzahl der Pumpe waren $8700$ Rpm, der Volumenstrom der Vergleichspumpe liegt unter Volllast bei $7.5 \,\unit{\frac{\liter}{\minute}}$. Mit der Querschnittsfläche des Rohrs von $0.025\, \unit{\centi \meter}^2$ kann so eine maximale Fließgeschwindigkeit von
$1.59 \, \unit{\frac{\meter}{\second}}$ angenommen werden.
Über den Dreisatz gilt es jetzt die Geschwindigkeiten für niedrigere Drehzahlen zu ermitteln.
Es ergeben sich folgende Werte 
\begin{align*}
     v_{7000\text{ Rpm}}&=1.2793\, \unit{\frac{\meter}{\second}}\\
     v_{6000\text{ Rpm}}&=1.0966\, \unit{\frac{\meter}{\second}}\\
     v_{5010\text{ Rpm}}&=0.9157\, \unit{\frac{\meter}{\second}}\\
     v_{4020\text{ Rpm}}&=0.7347\, \unit{\frac{\meter}{\second}}\\
     v_{2990\text{ Rpm}}&=0.5464\, \unit{\frac{\meter}{\second}}\, .\\
\end{align*}
\subsubsection{Bestimmung des Strömungsprofils}  
\label{sec:Profil}

Für den zweiten der Auswertung wird der Prismawinkel $\theta =15°$ verwendet. Die gemessenen Geschwindigkeiten und Streuintensität für 2 Pumpdrehzahlen sind in \autoref{tab:3920rpm} und \autoref{tab:6100rpm} aufgetragen.
Darunter sind die Geschwindigkeiten und Streuintensitäten der Drehzahlen je gegen die Messtiefe aufgetragen.
 \begin{table}
     \centering
     \caption{Streuintensität und Geschwindigkeit bei 3920 Umdrehungen pro Minute.}
     \begin{tabular}{c c c c c}
         \toprule
         Messtiefe $ \mathrm{/} \unit{\micro \second}$ &  Messtiefe $ \mathrm{/} \unit{\milli \meter}$ & $v \mathrm{/} \unit{\frac{\centi \meter}{\second}}$ & $v \mathrm{/} \unit{\frac{\meter}{\second}}$ & $I \mathrm{/} 1000\unit{\frac{V^2}{\second}}$\\
         \midrule
         12.0 &  -0.42 &  13.2 &  0.132 &  460\\
         12.5 &  0.33 &  11.5 &  0.115 &  1160\\
         13.0 &  1.08 &  13.2 &  0.132 &  1500\\
         13.5 &  1.83 &  14.8 &  0.148 &  1200\\
         14.0 &  2.58 &  14.8 &  0.148 &  1000\\
         14.5 &  3.33 &  16.5 &  0.165 &  1000\\
         15.0 &  4.08 &  16.5 &  0.165 &  1500\\
         15.5 &  4.83 &  16.5 &  0.165 &  1650\\
         16.0 &  5.58 &  16.5 &  0.165 &  1900\\
         16.5 &  6.33 &  16.5 &  0.165 &  1200\\
         17.0 &  7.08 &  16.5 &  0.165 &  750\\
         17.5 &  7.83 &  16.5 &  0.165 &  400\\
         18.0 &  8.58 &  16.5 &  0.165 &  350\\
         18.5 &  9.33 &  16.5 &  0.165 &  340\\
         19.0 &  10.08 &  14.8 &  0.148 &  320\\
         19.5 &  10.83 &  14.8 &  0.148 &  310\\
         \bottomrule
     \end{tabular}
     \label{tab:3920rpm}
\end{table}
\begin{table}
     \centering
     \caption{Streuintensität und Geschwindigkeit bei 6100 Umdrehungen pro Minute.}
     \begin{tabular}{c c c c c}
          \toprule
          Messtiefe $ \mathrm{/} \unit{\micro \second}$ &  Messtiefe $ \mathrm{/} \unit{\milli \meter}$ & $v \mathrm{/} \unit{\frac{\centi \meter}{\second}}$ & $v \mathrm{/} \unit{\frac{\meter}{\second}}$ & $I \mathrm{/} 1000\unit{\frac{V^2}{\second}}$\\
          \midrule
          12.0 &  -0.42 &  38.2 &  0.382 &  800\\
          12.5 &  0.33 &  44.6 &  0.446 &  1700\\
          13.0 &  1.08 &  50.9 &  0.509 &  2600\\
          13.5 &  1.83 &  57.3 &  0.573 &  3500\\
          14.0 &  2.58 &  66.8 &  0.668 &  3500\\
          14.5 &  3.33 &  76.4 &  0.764 &  3000\\
          15.0 &  4.08 &  82.7 &  0.827 &  2700\\
          15.5 &  4.83 &  82.7 &  0.827 &  2400\\
          16.0 &  5.58 &  82.7 &  0.827 &  2100\\
          16.5 &  6.33 &  73.2 &  0.732 &  1800\\
          17.0 &  7.08 &  70.0 &  0.700 &  1400\\
          17.5 &  7.83 &  66.8 &  0.668 &  1200\\
          18.0 &  8.58 &  66.8 &  0.668 &  1200\\
          18.5 &  9.33 &  66.8 &  0.668 &  1150\\
          19.0 &  10.08 &  70.0 &  0.700 &  1000\\
          19.5 &  10.83 &  73.2 &  0.732 &  800\\
          \bottomrule
     \end{tabular}
     \label{tab:6100rpm}
\end{table}

Der Nullpunkt der Messtiefe liegt hierbei auf der Oberfläche des Rohres.


\begin{figure}
     \centering
     \includegraphics[height = 6cm]{build/rpm39201.pdf}
     \caption{Geschwindigkeit aufgetragen gegen die Messtiefe bei 3920 Rpm.}
     \label{fig:39201}
\end{figure}

\begin{figure}
     \centering
     \includegraphics[height = 6cm]{build/rpm39202.pdf}
     \caption{Strömungsintensität aufgetragen gegen die Messtiefe bei 3920 Rpm.}
     \label{fig:39202}
\end{figure}

\begin{figure}
     \centering
     \includegraphics[height = 6cm]{build/rpm61001.pdf}
     \caption{Geschwindigkeit aufgetragen gegen die Messtiefe bei 6100 Rpm.}
     \label{fig:61001}
\end{figure}

\begin{figure}
     \centering
     \includegraphics[height = 6cm]{build/rpm61002.pdf}
     \caption{Strömungsintensität aufgetragen gegen die Messtiefe bei 6100 Rpm.}
     \label{fig:61002}
\end{figure}