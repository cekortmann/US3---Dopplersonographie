\section{Auswertung}
\label{sec:Auswertung}

\input{fehlerrechnung.tex}

\subsection{Bestimmung der Strömungsgeschwindigkeit}
\label{sec:Strömungsv}
Im ersten Teil der Auswertung wird die Strömungsgeschwindigkeit bestimmt. Die hierfür nötigen Messwerte sind in \autoref{tab:Stroemungsv} gelistet.
\begin{table}
    \centering
    \caption{Messwerte zu Bestimmung der Strömungsgeschwindigkeit.}
    \begin{tabular}{c c c c c c}
        \toprule
        Rpm der Pumpe  & Winkel $\theta \mathrm{/} \unit{\degree}$ & Winkel $\alpha \mathrm{/}  \unit{\degree}$ & $\nu_{\text{max}} \mathrm{/} \unit{\hertz}$ & $\nu_{\text{mean}} \mathrm{/} \unit{\hertz}$ &  $ \bigtriangleup\nu$ \\
        \midrule
        2990 & 15 & 80.06 & 175 & 85 & 90  \\
             & 30 & 70.53 & 210 & 110 & 100 \\
             & 60 & 54.74 & 303 & 159 & 144 \\
        4020 & 15 & 80.06 & 290 & 146 & 144 \\
             & 30 & 70.53 & 360 & 171 & 189 \\
             & 60 & 54.74 & 570 & 370 & 200 \\
        5010 & 15 & 80.06 & 380 & 183 & 197 \\
             & 30 & 70.53 & 565 & 269 & 296 \\
             & 60 & 54.74 & 960 & 513 & 447 \\
        6000 & 15 & 80.06 & 605 & 256 & 349 \\
             & 30 & 70.53 & 850 & 403 & 447 \\
             & 60 & 54.74 & 1505 & 757 & 748 \\
        7000 & 15 & 80.06 & 850 & 378 & 472  \\
             & 30 & 70.53 & 1160 & 519 & 641 \\
             & 60 & 54.74 & 2020 & 1013 & 1007 \\
        \bottomrule
    \end{tabular}
    \label{tab:Stroemungsv}
\end{table}
Eine Formel für dei Flussgeschwindigkeit ergibt sich durch das Umstellen von \autoref{eqn:dv} nach $v$
\begin{equation*}
v=\frac{c\cdot \bigtriangleup \nu}{2\nu_0\cdot \cos \alpha}\. .
\end{equation*}
So ergeben sich folgende Flussgeschwindigkeiten
\begin{table}
     \centering
     \caption{Fließgeschwindigkeiten der verschiedenen Winkel und Pumpleistungen.}
     \begin{tabular}{c c c c}
          \toprule
          Rpm der Pumpe & $v_{15°}\mathrm{/} \unit{\frac{\meter}{\second}}$ & $v_{30°}\mathrm{/} \unit{\frac{\meter}{\second}}$& $v_{60°}\mathrm{/} \unit{\frac{\meter}{\second}}$\\
          \midrule
          2990 & 0.2346 & 0.1350 & 0.1122\\
          4020 & 0.3754 & 0.2552 & 0.1559\\
          5010 & 0.5136 & 0.3996 & 0.3484\\
          6000 & 0.9098 & 0.6035 & 0.5831\\
          7000 & 1.2305 & 0.8654 & 0.7850\\ 
          \bottomrule
     \end{tabular}
\end{table}




\begin{table}
     \centering
     \caption{3920.}
     \begin{tabular}{c c c c c}
         \toprule
         Messtiefe $ \mathrm{/} \unit{\micro \second}$ &  Messtiefe $ \mathrm{/} \unit{\milli \meter}$ & $v \mathrm{/} \unit{\frac{\centi \meter}{\second}}$ & $v \mathrm{/} \unit{\frac{\meter}{\second}}$ & $I \mathrm{/} 1000\unit{\frac{V^2}{\second}}$\\
         \midrule
          
         \bottomrule
     \end{tabular}
     \label{tab:3920rpm}
 \end{table}