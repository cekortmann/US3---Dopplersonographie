\section{Aufbau}
\label{sec:Aufbau}

Die Messapparatur besteht aus einem Ultraschall Doppler-Generator im Pulsbetrieb sowie einer Ultraschallsonde mit einer Frequenz von $2\,\unit{\mega\hertz}$
und Untersuchungsobjekten. Die Ultraschallsonde dient dabei sowohl als Sender wie als Empfänger der Ultraschallwelle. 

Die Untersuchungsobjekte sind Strömungsrohre verschiedener Innen- und Außendurchmesser, welche mit einem Flüssigkeitsgemisch aus Wasser, Glycerin und 
Glaskugeln gefüllt ist. Die Strömungsgeschwindigkeit der Flüssigkeit wird im folgenden mit einer Zentrifugalpumpe eingestellt.

Um die Ankopplung der Ultraschallsonde an die Strömungsrohre zu erleichtern, werden Doppler-Prismen verwendet, die drei verschiedene Einfallswinkel besitzen, sodass 
der Winkel und der Abstand konstant gehalten werden kann. Die Dopplerprismen werden dabei auf das Rohrstücke gesetzt. Die Kontaktflächen werden mit einem Gel bedeckt, damit die Ultraschallwellen besser übertragen werden können.
