\section{Diskussion}
\label{sec:Diskussion}

In \autoref{sec:Strömungsv} ist der zu erwartende lineare Zusammenhang von $\frac{\Delta \nu}{\cos \alpha}$ mit der Strömungsgeschwindigkeit $v$. Ein Vergleich zu einem exakten Theoriewert bleibt jedoch aus, 
da aufgrund eines technischen Defekts die Durchflussrate der Pumpe nicht angezeigt werden konnte. Für die Geschwindigkeiten ergaben sich so folgende Abweichungen zu den theoretisch berechneten 
Geschwindigkeiten der anderen Pumpe von
\begin{align*}
    2990 \text{ Rpm}&:70.65\%\\
    4020 \text{ Rpm}&:64.32\%\\
    5010 \text{ Rpm}&:54.08\%\\
    6000 \text{ Rpm}&:36.27\%\\
    7000 \text{ Rpm}&:24.94\%\, .\\
\end{align*}
Um die Abweichung zu ermittelt wurde die Formel
\begin{equation*}
    \Delta x=\frac{x_{\text{exp}}-x_{\text{theo}}}{x_{\text{theo}}}
\end{equation*}
verwendet.
Die Abweichungen können daher kommen, dass die Pumpe keinen konstanten Druck halten konnte. Zudem haben die angezeigten Messwerte sehr stark geschwankt,
sodass genaues Ablesen unmöglich war. Auch die Ultraschallsonde hat sich während der Messung bewegt, was zu Messfehlern führt. Außerdem ist nicht 
auszuschließen, dass zu wenig Ultraschallgel zwischen Sonde und Prisma sowie zwischen Prisma und Strömungsrohr aufgetragen war, sodass diese nicht ordnungsgemäß gekoppelt
waren. 

Es ist auffällig, dass die Abweichungen mit einem höheren Druck abnehmen. Dies könnte daran liegen, dass die Glaskugeln bei einer niedrigen Geschwindigkeit
nicht so gut von der Flüssigkeit mitgerissen werden. Es ist auch möglich, dass bei einem niedrigen Druck nur wenige Glaskügelchen an der Stelle der 
Ultraschallsonde vorbeifließen.

In \autoref{sec:Profil} wird das Strömungsverhalten untersucht. Es war zu erwarten, dass sowohl die Strömungsgeschwindigkeit als auch die -intensität zunimmt, je weiter die Messtiefe in die Röhre eindringt.
Nach erreichen des Mittelpunktes ist zu erwarten, dass beide Messgrößen wieder abnehmen. Im Groben zeichnet sich auch dieser Trend bei den Messwerten ab. Jede der Kurven hat einen Peak bei bei einer im Rohr liegenden Messtiefe, welcher jedoch nicht ganz bei dem zu erwartenden Wert liegt.
Da das Rohr einen Außendurchmesser von $15\,$mm und einen Innendurchmesser von $10\,$ mm, ist der nach der Eichung zu erwartende Peak bei einer Messtiefe von $7.5\,$mm.
In den graphischen Darstellungen ist zu erkennen, dass dies nicht ganz der Fall ist. Grund hierfür könnte die Größe der Kugeln selbst sein. Je nachdem in welcher Tiefe die Glaskugel getroffen wurde, kann es so zu unterschiedlichen Werten kommen. 
Hinzu kommen die bereits oben erwähnten Probleme und Fehlerquellen.
