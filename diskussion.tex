\section{Diskussion}
\label{sec:Diskussion}

In \autoref{sec:Strömungsv} ist der zu erwartende lineare Zusammenhang von $\frac{\Delta \nu}{\cos \alpha}$ mit der Strömungsgeschwindigkeit $v$. Ein Vergleich zu einem Theoriewert bleibt jedoch aus, 
da aufgrund eines technischen Defekts die Durchflussrate der Pumpe nicht angezeigt werden konnte. Den Graphen ist jedoch zu entnehmen, dass trotz der Verwackelungen, die durch das Halten der Sonde entstanden ist, und 
der Tatsache, dass sich die Drehzahl der Pumpe nach einer Weile automatisch erhöht hat, die Messwerte relativ frei von Messungenauigkeiten sind. DIe Messwerte liegen alle auf der jeweiligen Ausgleichsgeraden.

In \autoref{sec:Profil} wird das Strömungsverhalten untersucht. Es war zu erwarten, dass sowohl die Strömungsgeschwindigkeit als auch die -intensität zunimmt, je weiter die Messtiefe in die Röhre eindringt.
Nach erreichen des Mittelpunktes ist zu erwarten, dass beide Messgrößen wieder abnehmen. Im Groben zeichnet sich auch dieser Trend bei den Messwerten ab. Jede der Kurven hat einen Peak bei bei einer im Rohr liegenden Messtiefe, welcher jedoch nicht ganz bei dem zu erwartenden Wert liegt.
Da das Rohr einen Außendurchmesser von $15\,$mm und einen Innendurchmesser von $10\,$ mm, ist der nach der Eichung zu erwartende Peak bei einer Messtiefe von $7.5\,$mm.
In den graphischen Darstellungen ist zu erkennen, dass dies nicht ganz der Fall ist. Grund hierfür könnte die Größe der Kugeln selbst sein. Je nachdem in welcher Tiefe die Glaskugel getroffen wurde, kann es so zu unterschiedlichen Werten kommen. 
Hinzukommen die bereits oben erwähnten Probleme und Fehlerquellen.
