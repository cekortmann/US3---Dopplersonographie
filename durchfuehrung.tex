\section{Durchführung}
\label{sec:Durchführung}

Der Versuch besteht aus zwei Teilversuchen. Zuerst wird die Strömungsgeschwindigkeit als Funktion des Dopplerwinkels $\Theta$ für fünf verschiedene
Flussgeschwindigkeiten gemessen. Dafür wird an der Zentrifugalpumpe eine Geschwindigkeit eingestellt; das Sample Volume des Ultraschallgenerators 
muss auf LARGE stehen. Nun werden mittels der Ultraschallsonde für jede Geschwindigkeit die Frequenzverschiebung $\symup{\Delta} \nu$ gemessen.

Im zweiten Versuchsteil wird das Strömungsprofil der Doppler-Flüssigkeit an einem 3/8-Schlauch mit einem Dopplerwinkel von 15° gemessen. Dazu muss zuerst 
das Sample Volume auf SMALL umgestellt werden. Mit dem Regler DEPTH lässt sich die Messtiefe einstellen. Bei einer maximalen Pumpleistung von 79\% 
wird nun die Strömungsgeschwindigkeit und deer Streuintensitätswert gemessen. Begonnen bei einer Messtiefe von $30\,\unit{\milli\meter}$ bis zu einer 
Messtiefe von $11\,\unit{\milli\meter}$ werden die Werte in $0.75\,\unit{\milli\meter}$-Schritten aufgenommen. Die gesamte Messung wird für eine 
Pumpleistung von 45\% wiederholt.